\documentclass[10pt,a4paper,ngerman]{article}
% ,twocolumn
\usepackage[ngerman]{babel}
\usepackage[T1]{fontenc}
\usepackage[utf8]{inputenc}

\usepackage{graphicx}

% against underfull boxes
\usepackage{microtype}

\usepackage[a4paper, right=2cm, top=3cm, left=3cm, bottom=3cm]{geometry}

% \usepackage{cite}
% \usepackage{csquotes}
% \usepackage[
%     style=numeric,
%     citestyle=numeric
% ]{biblatex}
% \addbibresource{references.bib} %Imports bibliography file


% \usepackage{proposal}

\usepackage{fancyhdr}
\pagestyle{fancy}
\fancyhf{}
% \rhead{Proposal zur Bachelorarbeit}
% \lhead{Technische Hochschule Nürnberg}


\title{\textbf{Analyse von Wifi Probe-Requests zur personenbezogenen Anwesenheitserkennung}}
\author{Lennart Heimbs, Bachelorstudium Medizintechnik}
% \betreuer{Prof.-Dr. Oliver Hofmann, Betreuender Dozent der Technischen Hochschule Nürnberg}

\begin{document}
    \maketitle

    \section*{Grundlage}
    \begin{itemize}
        \item WLAN fähige Geräte haben die Eigenschaft sich automatisch mit bekannten WLAN-Netzen zu Verbinden
        \begin{itemize}
           \item Nutzung von Probe-Requests
        \end{itemize}
        \item Probe-Requests sind Kontrollframes von WLAN fähigen Geräten, die periodisch gesendet werden, um bekannte WLAN-Netzwerke zu finden und sich automatisch mit diesenzu verbinden
        \item enthalten die MAC-Adresse des sendenden Geräts und zum Teil auch die SSID von bereits bekannten WLAN-Netzen
    \end{itemize}
    \includegraphics[scale=0.4]{/home/lenny/Pictures/Screenshots/Screenshot_2020-11-20_probe_reqs.png}
    \\

    \section*{Ziele}
    \begin{itemize}
        \item Aufnahme Probe-Requests mit Raspberry Pi
        \item Analyse der aufgenommenen Requests anhand von Algorithmen aus der Literatur
        \item Untersuchung bezüglich Zuverlässigkeit der Anwesenheitserkennung
        \item Prototypische implementierung einer Anwendung zur Anwesenheitserkennung in der Hochschule
    \end{itemize}

    \section*{Aktuelles Arbeitspaket}
    \begin{itemize}
        \item Verbesserung der Aufnahme der Probe-Requests
        \item Vorläufige analyse der bereits aufgenommenen Daten
    \end{itemize}

    \section*{Nächster Arbeitsschritt}
    \begin{itemize}
        \item Implementierung der Analysealgorithmen
    \end{itemize}

\end{document}