\section{Vorgehensweise}\label{approach}
Ausgehend von der Einarbeitung in das Thema soll zunächst ein Prototyp zur Sammlung der Probe-Requests eingerichtet werden.
Dabei soll der Aufbau der Kontrollframes untersucht werden und es soll festgelegt werden welche Daten für die Anwesenheitserkennung relevant sind.
Es wird eine Datenbank verwendet, um die gesammelten Requests zur weiteren Verarbeitung bereit zu stellen.
\\

Anhand einer Literaturrecherche werden anschließend Algorithmen und Methoden zur Anwesenheitsanalyse gewählt.
Diese sollen prototypisch implementiert und experimentell hinsichtlich der Zuverlässigkeit und der Identifikationsmöglichkeit von Personen verglichen werden.
Zur Demonstration der evaluierten Methoden wird eine Anwendung entwickelt, die die Aufgabe eines Anwesenheitsblattes übernimmt und anhand einstellbarer Metriken erkennen kann, ob bestimmte Personen anwesend sind.
\\

Abschließend wird die prototypische Implementierung basierend auf den in \nameref{goal} genannten Fragestellungen evaluiert und bewertet.