\section{Zielsetzung}\label{goal}

Das Ziel dieser Arbeit ist die Eignung von Wifi Probe Requests in der Anwesenheitserfassung zu untersuchen.
Im Vergleich mit herkömmlichen Systemen wie dem Scannen von QR-Codes oder einer Unterschriftenliste soll erforscht werden ob die Analyse von Probe Requests erlaubt, die Anwesenheit individueller Peronen zu erfassen.
\\

Dabei steht die eindeutige Identifizierung der Personen im Vordergrund:
Ein "False-Positive" Fall, in dem eine abwesende Person als anwesend erkannt wird, ist zu vermeiden, da die Behebung eines solchen Fehlers Wissen über die tatsächliche Anwesenheit vorraussetzt, welches von eben dieser Anwendung bereitgestellt werden soll.
Das Nicht-Erkennen von anwesenden Personen ist zu minimieren, aber weniger kritisch als das Erkennen von Nicht-Anwesenden Personen, da diese ihre Anwesenheit im Falle eines Fehlers eigenmächtig nachtragen können.
